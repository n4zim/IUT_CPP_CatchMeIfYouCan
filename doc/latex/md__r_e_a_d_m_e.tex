Il faut installer la librairie S\-F\-M\-L 1.\-6 pour pouvoir compiler. Exemple de commande d'installation pour ubuntu 13.\-10 \-:

sudo apt-\/get install mesa-\/common-\/dev ibglu-\/dev libsfml-\/dev

\section*{S\-O\-U\-N\-D C\-R\-E\-D\-I\-T}

Validation sound effect by broumbroum, under a creative commons 3.\-0 by licence \href{http://www.freesound.org/people/broumbroum/sounds/50565/}{\tt http\-://www.\-freesound.\-org/people/broumbroum/sounds/50565/}

Musique du jeu \char`\"{}\-Trickyness\char`\"{} par Josua Gonzalez 